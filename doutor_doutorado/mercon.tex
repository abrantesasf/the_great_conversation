%%%%%%%%%%%%%%%%%%%%%%%%%%%%%%%%%%%%%%%%%%%%%%%%%%%%%%%%%%%%%%%%%%%%%%%%%%%%%%%%%
% Template: Article
%
% Por: Abrantes Araújo Silva Filho
%      abrantesasf@gmail.com
%
% Citação: Se você gostou deste template, por favor ajude a divulgá-lo mantendo
%          o link para meu repositório GitHub em:
%          https://github.com/abrantesasf/LaTeX
%%%%%%%%%%%%%%%%%%%%%%%%%%%%%%%%%%%%%%%%%%%%%%%%%%%%%%%%%%%%%%%%%%%%%%%%%%%%%%%%%




%%%%%%%%%%%%%%%%%%%%%%%%%%%%%%%%%%%%%%%%%%%%%%%%%%%%%%%%%%%%%%%%%%%%%%%%%%%%%%%%%
%%% Configura o tipo de documento, papel, tamanho da fonte e informações básicas
%%% para as proriedades do PDF/DVIPS e outras propriedades do documento
\RequirePackage{ifpdf}
\ifpdf
  % Classe, língua e tamanho da fonte padrão. Outras opções a considerar:
  %   draft
  %   onecolumn (padrão) ou twocolumn (OU usar o package multicol)
  %   fleqn com ou sem leqno (alinhamento à esquerda das fórmulas e dos números)
  %   oneside (padrão para article ou report) ou twoside (padrão para book)
  \documentclass[pdftex, brazil, 12pt, oneside]{article}
\else
  % Classe, língua e tamanho da fonte padrão. Outras opções a considerar:
  %   draft
  %   onecolumn (padrão) ou twocolumn (OU usar o package multicol)
  %   fleqn com ou sem leqno (alinhamento à esquerda das fórmulas e dos números)
  %   oneside (padrão para article ou report) ou twoside (padrão para book)
  \documentclass[brazil, 12pt]{article}
\fi


%%%%%%%%%%%%%%%%%%%%%%%%%%%%%%%%%%%%%%%%%%%%%%%%%%%%%%%%%%%%%%%%%%%%%%%%%%%%%%%%%
%%% Carrega pacotes iniciais necessários para estrutura de controle e para a
%%% criação e o parse de novos comandos
\usepackage{ifthen}
\usepackage{xparse}


%%%%%%%%%%%%%%%%%%%%%%%%%%%%%%%%%%%%%%%%%%%%%%%%%%%%%%%%%%%%%%%%%%%%%%%%%%%%%%%%%
%%% Configuração do tamanho da página, margens, espaçamento entrelinhas e, se
%%% necessário, ativa a indentação dos primeiros parágrafos.
\ifpdf
  \usepackage[pdftex]{geometry}
\else
  \usepackage[dvips]{geometry}
\fi
\geometry{a4paper, left=2.0cm, right=2.0cm, top=2.0cm, bottom=2.0cm}

\usepackage{setspace}
  %\singlespacing
  \onehalfspacing
  %\doublespacing


%%%%%%%%%%%%%%%%%%%%%%%%%%%%%%%%%%%%%%%%%%%%%%%%%%%%%%%%%%%%%%%%%%%%%%%%%%%%%%%%%
%%% Configurações de cabeçalho e rodapé:
\usepackage{fancyhdr}
%\setlength{\headheight}{1cm}
%\setlength{\footskip}{1.5cm}
%\renewcommand{\headrulewidth}{0.3pt}
%\renewcommand{\footrulewidth}{0.0pt}
%\pagestyle{fancy}
%\renewcommand{\sectionmark}[1]{%
%  \markboth{\uppercase{#1}}{}}
%\renewcommand{\subsectionmark}[1]{%
%  \markright{\uppercase{\thesubsection \hspace{0.1cm} #1}}{}}
%\fancyhead{}
%\fancyfoot{}
%\newcommand{\diminuifonte}{%
%    \fontsize{9pt}{9}\selectfont
%}
%\newcommand{\aumentafonte}{%
%    \fontsize{12}{12}\selectfont
%}
% Configura cabeçalho e rodapé para documentos TWOSIDE
%\fancyhead[EL]{\textbf{\thepage}}
%\fancyhead[EC]{}
%\fancyhead[ER]{\diminuifonte \textbf{\leftmark}}
%\fancyhead[OR]{\textbf{\thepage}}
%\fancyhead[OC]{}
%\fancyhead[OL]{\diminuifonte \textbf{\rightmark}}
%\fancyfoot[EL,EC,ER,OR,OC,OL]{}
% Configura cabeçalho e rodapé para documentos ONESIDE
%\lhead{ \fancyplain{}{sup esquerdo} }
%\chead{ \fancyplain{}{sup centro} }
%\rhead{ \fancyplain{}{\thesection} }
%\lfoot{ \fancyplain{}{inf esquerdo} }
%\cfoot{ \fancyplain{}{inf centro} }
%\rfoot{ \fancyplain{}{\thepage} }


%%%%%%%%%%%%%%%%%%%%%%%%%%%%%%%%%%%%%%%%%%%%%%%%%%%%%%%%%%%%%%%%%%%%%%%%%%%%%%%%%
%%% Configurações de encoding, lingua e fontes:
\usepackage[T1]{fontenc}
\usepackage[utf8]{inputenc}
\usepackage{babel}

% Altera a fonte padrão do documento (nem todas funcionam em modo math):
%   phv = Helvetica
%   ptm = Times
%   ppl = Palatino
%   pbk = bookman
%   pag = AdobeAvantGarde
%   pnc = Adobe NewCenturySchoolBook
\renewcommand{\familydefault}{ppl}


%%%%%%%%%%%%%%%%%%%%%%%%%%%%%%%%%%%%%%%%%%%%%%%%%%%%%%%%%%%%%%%%%%%%%%%%%%%%%%%%%
%%% Carrega pacotes para referências cruzadas, citações dentro do documento,
%%% links para internet e outros.Configura algumas opções.
%%% Não altere a ordem de carregamento dos packages.
\usepackage{varioref}
\ifpdf
  \usepackage[pdftex]{hyperref}
    \hypersetup{
      % Informações variáveis em cada documento (MUDE AQUI!):
      pdftitle={Doutor é quem tem doutorado?},
      pdfauthor={Paulo Roberto Merçon de Vargas},
      pdfsubject={Discussão sobre o uso da palavra doutor para designar médicos},
      pdfkeywords={doutor, doutorado, etimologia, história},
      pdfinfo={
        CreationDate={}, % Ex.: D:AAAAMMDDHH24MISS
        ModDate={}       % Ex.: D:AAAAMMDDHH24MISS
      },
      % Coisas que você não deve alterar se não souber o que está fazendo:
      unicode=true,
      pdflang={pt-BR},
      bookmarksopen=true,
      bookmarksnumbered=true,
      bookmarksopenlevel=5,
      pdfdisplaydoctitle=true,
      pdfpagemode=UseOutlines,
      pdfstartview=FitH,
      pdfcreator={LaTeX with hyperref package},
      pdfproducer={pdfTeX},
      pdfnewwindow=true,
      colorlinks=true,
      citecolor=green,
      linkcolor=red,
      filecolor=cyan,
      urlcolor=blue
    }
\else
  \usepackage{hyperref}
\fi
\usepackage{cleveref}
\usepackage{url}


%%%%%%%%%%%%%%%%%%%%%%%%%%%%%%%%%%%%%%%%%%%%%%%%%%%%%%%%%%%%%%%%%%%%%%%%%%%%%%%%%
%%% Carrega bibliotecas de símbolos (matemáticos, físicos, etc.), fontes
%%% adicionais, e configura algumas opções
\usepackage{amsmath}
\usepackage{amssymb}
\usepackage{amsfonts}
\usepackage{siunitx}
  \sisetup{group-separator = {.}}
  \sisetup{group-digits = {false}}
  \sisetup{output-decimal-marker = {,}}
\usepackage{bm}
\usepackage{cancel}
% Altera separador decimal via comando, se necessário (prefira o siunitx):
%\mathchardef\period=\mathcode`.
%\DeclareMathSymbol{.}{\mathord}{letters}{"3B}
  

%%%%%%%%%%%%%%%%%%%%%%%%%%%%%%%%%%%%%%%%%%%%%%%%%%%%%%%%%%%%%%%%%%%%%%%%%%%%%%%%%
%%% Carrega packages relacionados à computação
\usepackage{algorithm2e}
\usepackage{algorithmicx}
\usepackage{algpseudocode}
\usepackage{listings}
  \lstset{literate=
    {á}{{\'a}}1 {é}{{\'e}}1 {í}{{\'i}}1 {ó}{{\'o}}1 {ú}{{\'u}}1
    {Á}{{\'A}}1 {É}{{\'E}}1 {Í}{{\'I}}1 {Ó}{{\'O}}1 {Ú}{{\'U}}1
    {à}{{\`a}}1 {è}{{\`e}}1 {ì}{{\`i}}1 {ò}{{\`o}}1 {ù}{{\`u}}1
    {À}{{\`A}}1 {È}{{\'E}}1 {Ì}{{\`I}}1 {Ò}{{\`O}}1 {Ù}{{\`U}}1
    {ä}{{\"a}}1 {ë}{{\"e}}1 {ï}{{\"i}}1 {ö}{{\"o}}1 {ü}{{\"u}}1
    {Ä}{{\"A}}1 {Ë}{{\"E}}1 {Ï}{{\"I}}1 {Ö}{{\"O}}1 {Ü}{{\"U}}1
    {â}{{\^a}}1 {ê}{{\^e}}1 {î}{{\^i}}1 {ô}{{\^o}}1 {û}{{\^u}}1
    {Â}{{\^A}}1 {Ê}{{\^E}}1 {Î}{{\^I}}1 {Ô}{{\^O}}1 {Û}{{\^U}}1
    {œ}{{\oe}}1 {Œ}{{\OE}}1 {æ}{{\ae}}1 {Æ}{{\AE}}1 {ß}{{\ss}}1
    {ű}{{\H{u}}}1 {Ű}{{\H{U}}}1 {ő}{{\H{o}}}1 {Ő}{{\H{O}}}1
    {ç}{{\c c}}1 {Ç}{{\c C}}1 {ø}{{\o}}1 {å}{{\r a}}1 {Å}{{\r A}}1
    {€}{{\euro}}1 {£}{{\pounds}}1 {«}{{\guillemotleft}}1
    {»}{{\guillemotright}}1 {ñ}{{\~n}}1 {Ñ}{{\~N}}1 {¿}{{?`}}1
  }
  

%%%%%%%%%%%%%%%%%%%%%%%%%%%%%%%%%%%%%%%%%%%%%%%%%%%%%%%%%%%%%%%%%%%%%%%%%%%%%%%%%
%%% Ativa suporte extendido a cores
\usepackage[svgnames]{xcolor} % Opções de cores: usenames (16), dvipsnames (64),
                              % svgnames (150) e x11names (300).


%%%%%%%%%%%%%%%%%%%%%%%%%%%%%%%%%%%%%%%%%%%%%%%%%%%%%%%%%%%%%%%%%%%%%%%%%%%%%%%%%
%%% Suporte à importação de gráficos externos
\ifpdf
  \usepackage[pdftex]{graphicx}
\else
  \usepackage[dvips]{graphicx}
\fi


%%%%%%%%%%%%%%%%%%%%%%%%%%%%%%%%%%%%%%%%%%%%%%%%%%%%%%%%%%%%%%%%%%%%%%%%%%%%%%%%%
%%% Suporte à criação de gráficos proceduralmente na LaTeX:
\usepackage{tikz}
  \usetikzlibrary{arrows,automata,backgrounds,matrix,patterns,positioning,shapes,shadows}


%%%%%%%%%%%%%%%%%%%%%%%%%%%%%%%%%%%%%%%%%%%%%%%%%%%%%%%%%%%%%%%%%%%%%%%%%%%%%%%%%
%%% Packages para tabelas
\usepackage{array}
\usepackage{longtable}
\usepackage{tabularx}
\usepackage{tabu}
\usepackage{lscape}
\usepackage{colortbl}  
\usepackage{booktabs}


%%%%%%%%%%%%%%%%%%%%%%%%%%%%%%%%%%%%%%%%%%%%%%%%%%%%%%%%%%%%%%%%%%%%%%%%%%%%%%%%%
%%% Packages ambientes de listas
\usepackage{enumitem}
\usepackage[ampersand]{easylist}


%%%%%%%%%%%%%%%%%%%%%%%%%%%%%%%%%%%%%%%%%%%%%%%%%%%%%%%%%%%%%%%%%%%%%%%%%%%%%%%%%
%%% Packages para suporte a ambientes floats, captions, etc.:
\usepackage{float}
\usepackage{wrapfig}
\usepackage{placeins}
\usepackage{caption}
\usepackage{sidecap}
\usepackage{subcaption}


%%%%%%%%%%%%%%%%%%%%%%%%%%%%%%%%%%%%%%%%%%%%%%%%%%%%%%%%%%%%%%%%%%%%%%%%%%%%%%%%%
%%% Meus comandos específicos:
% Commando para ``italizar´´ palavras em inglês (e outras línguas!):
\newcommand{\ingles}[1]{\textit{#1}}

% Commando para colocar o espaço correto entre um número e sua unidade:
\newcommand{\unidade}[2]{\ensuremath{#1\,\mathrm{#2}}}
\newcommand{\unidado}[2]{{#1}\,{#2}}

% Produz ordinal masculino ou feminino dependendo do segundo argumento:
\newcommand{\ordinal}[2]{%
#1%
\ifthenelse{\equal{a}{#2}}%
{\textordfeminine}%
{\textordmasculine}}


%%%%%%%%%%%%%%%%%%%%%%%%%%%%%%%%%%%%%%%%%%%%%%%%%%%%%%%%%%%%%%%%%%%%%%%%%%%%%%%%%
%%% Hifenização específica quando o LaTeX/Babel não conseguirem hifenizar:
\babelhyphenation{Git-Hub}
\usepackage{exsol}


%%%%%%%%%%%%%%%%%%%%%%%%%%%%%%%%%%%%%%%%%%%%%%%%%%%%%%%%%%%%%%%%%%%%%%%%%%%%%%%%%
%%%%%%%%%%%%%%%%%%%%%%%%%%%%%%%%%%%%%%%%%%%%%%%%%%%%%%%%%%%%%%%%%%%%%%%%%%%%%%%%%
%%%%%%%%%%%%%%%%%%%%%%%%%%%%%%%%%%%%%%%%%%%%%%%%%%%%%%%%%%%%%%%%%%%%%%%%%%%%%%%%%
%%%%%%%%%%%%%%%%%%%%%%%%%%%%%%%%%%%%%%%%%%%%%%%%%%%%%%%%%%%%%%%%%%%%%%%%%%%%%%%%%
%%%%%%%%%%%%%%%%%%%%%%%%%%%%%% COMEÇA O DOCUMENTO %%%%%%%%%%%%%%%%%%%%%%%%%%%%%%%
%%%%%%%%%%%%%%%%%%%%%%%%%%%%%%%%%%%%%%%%%%%%%%%%%%%%%%%%%%%%%%%%%%%%%%%%%%%%%%%%%
%%%%%%%%%%%%%%%%%%%%%%%%%%%%%%%%%%%%%%%%%%%%%%%%%%%%%%%%%%%%%%%%%%%%%%%%%%%%%%%%%
%%%%%%%%%%%%%%%%%%%%%%%%%%%%%%%%%%%%%%%%%%%%%%%%%%%%%%%%%%%%%%%%%%%%%%%%%%%%%%%%%
%%%%%%%%%%%%%%%%%%%%%%%%%%%%%%%%%%%%%%%%%%%%%%%%%%%%%%%%%%%%%%%%%%%%%%%%%%%%%%%%%
\begin{document}
\title{Doutor é quem tem doutorado?}
\author{Paulo Roberto Merçon de Vargas}
\date{\small{Vitória, 24 de janeiro de 2002}}
\maketitle
\renewcommand{\abstractname}{Atenção! Entenda o título deste documento:}
\abstract{O texto abaixo é uma cópia do discurso do Prof.\ Dr.\ \textbf{Paulo Roberto Merçon
  de Vargas}, proferido em 24 de janeiro de 2002, para os formandos da \ordinal{58}{a}
  Turma de Medicina da Universidade Federal do Espírito Santo (UFES). Note que \emph{o discurso
  original não tem título} e a designação de ``Doutor é quem tem doutorado?''\ como
  título foi feita por mim, Dr.\ Abrantes Araújo Silva Filho, como forma de chamar
  a atenção para o movimento que questiona o uso de ``doutor'' como forma de
  tratamento para os médicos. Agradeço ao Dr.\ Leonardo Goltara, que manteve a
  cópia original do discurso em seu
  \href{http://vicetreco.blogspot.com/2013/04/a-maior-aula-de-medicina-de-todas.html}{blog pessoal}.
  Este documento está disponível para download em um
  \href{https://github.com/abrantesasf/the\_great\_conversation}{repositório no GitHub}.}
%\tableofcontents
%\newpage

\vspace {0.5 cm}

\paragraph{Discurso do Paraninfo Paulo Roberto Merçon de Vargas, proferido em
24 de janeiro de 2002, para a 58\textordfeminine\ Turma de Medicina da UFES.}\ \\%

Magnífico Reitor da UFES, Prof.\ José Weber Freire Macedo; Magnífico Vice-Reitor da UFES,
Prof.\ Rubens Sérgio Rasseli; Excelentíssimo Sr.\ Diretor do Centro Biomédico,
Prof.\ Wilson Mário Zanotti; Excelentíssimos Patrono, Professores e demais homenageados
da turma; Senhores pais, esposos, esposas e demais parentes dos doutorandos; Prezados senhores
e senhoras; Queridos doutorandos\dots

Hoje é um dia de alegria para todos nós. Cada um de vocês, doutorandos, familiares,
professores e amigos, têm vários motivos para estarem felizes com esta formatura.
Também eu.

Para mim, ser escolhido paraninfo foi uma honra e uma alegria. É ótimo ser reconhecido
e homenageado, mesmo que isto implique em fazer um discurso --- alguns formandos me disseram
exatamente isto: que me escolheram para ouvir um discurso! Muito bem, vocês o terão. Eu
espero que, em cumprimento do papel de paraninfo, este discurso seja um auxílio para a
iniciação na vida profissional de vocês. Não será um discurso longo: uma pequena aula
e um pedido.

Em primeiro lugar, a aula.

Não se preocupem: será uma aula breve. Eu gostaria de incluir uma prova de 200 questões
do tipo marcar certo ou errado. Com uma resposta errada anulando uma resposta certa, mas,
pensando bem, eu teria que corrigir prova e isto é muito chato. Desisto da prova.
Além do mais, vocês estão formando, provas ficaram para trás!

Minha intenção é que seja uma aula para todos (como o foi para mim). Não só para os formandos.
Não pretendo ser exaustivo e advirto que conheço menos do que deveria sobre o tema.
Mas dar aula ou ensinar é apenas um outro nome para aprender!

O tema da aula resume-se em uma palavra: a palavra ``doutor''.

Hoje, vocês receberão o grau de doutor, um reconhecimento universitário de competência
para exercer a Medicina. É um privilégio e uma responsabilidade. Mas porque será, mesmo,
que nos chamamos ``doutores''? E por que nossos pacientes há muitos séculos, nos chamam
de doutor? Sobre isto trata a aula.

A palavra doutor tem vários significados. O novo dicionário Houaiss lista 15 usos ou
sentidos para esta palavra! Os três sentidos mais básicos são: 1) o sentido mais antigo
da palavra doutor é ``um homem sábio''; 2) o segundo mais antigo é ``um médico''; e
3) no sentido mais recente (em uso desde o século XII DC) designa ``o maior grau universitário''
(e aqui se incluem os graus de doutor conferidos a não médicos).

Hoje está em moda, ou se pretende que assim seja feito, limitar o uso ao sentido ``3'',
ao possuidor do título universitário de doutor. Todavia, uma análise histórica dos usos
deste termo revela que todos derivam de uma só fonte. Vejamos qual seja.

Há séculos, duas palavras têm sido empregadas para designar o profissional da Medicina, isto é,
aquele que pratica a arte de curar: \ingles{physician} e doutor. Desculpem-me, mas é preciso usar
esta palavra inglesa, porque não há correspondente em nossa língua. Nos países de língua inglesa,
formalmente, designa-se o médico como \ingles{physician} (que não podemos traduzir por físico),
mas em linguagem comum, popular e como vocativo diz-se \ingles{doctor}, doutor. O paciente chama
seu médico de \ingles{doctor} (coloquialmente, \ingles{doc}). Menciono esta particularidade da
língua inglesa porque ela conserva em uso atual ambas as palavras que descrevem a atividade e o
profissional médico. Nos países de herança latina, como o Brasil, falamos sempre doutor.

Qual a história destas palavras? O que tem a ver conosco e com esta cerimônia de colação de grau?
Bem, vocês agora são médicos e vão receber o grau de doutor. Serão doutores. É preciso saber o
que isto significa! De onde vêm estas duas palavras? Porquê são usadas? Qual o significado
original? Sim, porque estas palavras carregam, até nossos dias, seu significado original.
E, eu penso, esta compreensão é importante para todos nós, pacientes e médicos. É importante,
especialmente para os doutorandos, hoje.

Cada uma delas vem de uma das duas grandes heranças que culminaram na civilização ocidental:
a herança judáico-cristã e a herança grega. Pretendo comparar estas heranças e mostrar
que estão conosco, em nosso dia-a-dia.

A palavra \ingles{physician} é uma herança grega, especialmente importante nos escritos de
Aristóteles, o filósofo, que viveu no século III antes de Cristo. Deriva de \ingles{physis}, que
designa todas as coisas que conhecemos ou acreditamos e, neste sentido, corresponde à palavra
latina natureza.

Na Medicina Grega (e, depois, na Medicina Romana), a cura das doenças se operava por Esculápio,
o deus da cura ou o deus da Medicina. Seria excessivo explicar aqui toda a noção grega de
causa e mecanismo de doença. Vou me limitar ao papel do médico.
O médico era o agente do deus
da cura. Alguém iniciado, a quem eram revelados os mistérios e o poder de controlar a
\ingles{phisys}, a natureza. Em razão disto, deste poder de controle, o médico era considerado
um ser especial, um homem mais que homem, um semideus. Neste papel, a ênfase estava no controle,
no poder sobre a natureza e sobre o homem. Toda a responsabilidade estava no médico: ele era
o agente da cura. --- Eu acho que deve advir disto a acusação (às vezes justificada), que os
médicos se sentem ou gostam de posar como ``deuses''!

Prezados doutorandos, vocês se recordam que, na sala de aula de Patologia, havia um lugar que
chamávamos Olimpo, um lugar onde ficavam os deuses? Era uma brincadeira, claro. Mas
brincadeiras e piadas são excelentes para falar coisas sérias.

Vejamos como funcionava o \ingles{physician}, o médico grego. Ele intercedia e procurava predispor
Esculápio a favor do doente através de oferendas e também recomendava tratamentos em seu nome.
Esculápio, como deus, podia, sempre, curar todas as doenças. Em outras palavras, o doente obtinha
a cura do deus Esculápio, através do médico. O doente não conhecia nada, nada lhe era ensinado,
nem tinha acesso direto ao deus da cura. Era um agente apenas passivo.

Mas o que acontecia se algo saia errado, se a doença seguia seu curso ou se o doente morria?
De quem era a culpa? Do médico, claro. Ninguém atribuía a culpa a Esculápio (afinal, deus não erra).
Ora, isto está se tornando moda nos dias de hoje: a culpa sempre é atribuída ao médico! Não é por acaso.

Nesta situação, como podiam se defender os médicos? Fácil! Examinado o doente e diagnosticado um câncer,
o médico recusava-se a tratar do paciente! --- Parece que Esculápio não tinha a eficácia dos oncologistas
de hoje!

Não se iludam, idéias têm conseqüências: ainda no século XIX, haviam médicos que assim procediam!
Outra forma do médico se defender era fazer a culpa recair no paciente: você não quer ser curado,
você não pode pagar as oferendas, você não tem fé, etc. Em outras palavras, evitava-se o risco e
a responsabilidade. Hoje, podemos dizer, evita-se o processo por má prática.

Não desejo para mim, nem para vocês, doutorandos, meus novos colegas, uma Medicina destas!
No entanto, isto não soa familiar a vocês? Não nos parece, às vezes, que estamos vivendo justamente
isto! Eu acho que sim. Os gregos esperavam muito dos seus médicos. Nossos doentes esperam muito de nós.

É verdade que a Medicina progrediu muito. A mídia está, sempre, repleta de ``milagres'' médicos.
É verdade que conhecemos muito e temos recursos terapêuticos inimagináveis há 100 anos atrás.
Mas não podemos tudo. Não somos deuses!

Isto quanto à palavra \ingles{physician} e à Medicina Grega. E quanto à palavra doutor, que nos
interessa mais diretamente? Lembrem-se, vocês, a partir de hoje, são doutores!

A palavra doutor vem do latim \ingles{docere}, que significa ensinar ou ensinador, professor.
Tem o mesmo sentido das palavras hebraicas para Mestre e Rabbi. Seu uso remonta aos tempos do
Velho Testamento, cerca do século VII antes de Cristo.

O Novo Testamento, no evangelho de Lucas, diz-nos que, tendo Jesus se separado de seus pais,
durante a festa, em Jerusalém, estes o encontraram no meio dos doutores, ensinando.
Nosso Salvador era um doutor.

Na tradição judaico-cristã, um doutor é um homem sábio, um entendido na doutrina, isto é,
nas leis religiosas, uma autoridade em Teologia. Por isso, muito cedo, na história da Igreja Cristã,
os professores de doutrina (especialmente os mais notáveis) foram chamados doutores.
Mas, como vieram os médicos a se igualar a estes professores religiosos? Porque os médicos
passaram a ser chamados de doutores?

Para entender isto precisamos estudar outra palavra: a palavra salvação. Em latim, salvação
tem o sentido de saúde, bem estar, prosperidade, conservação da vida. Na tradição judaico-cristã,
um doutor se ocupa da salvação, isto é, da saúde. Saúde total, saúde do corpo e saúde da alma.

Para a fé bíblica, o cuidado do corpo faz parte da salvação, a saúde é um aspecto da salvação.
Recordem-se que, ainda hoje, alguns pacientes nos perguntam: ``E ai, doutor, tem salvação?''
Ou então: ``Vai se salvar, doutor?'' O médico que busca conhecer, que ensina a cuidar do corpo,
que ajuda a preservar e a restaurar a saúde, tem um chamado sacerdotal, tal qual o ministro
religioso (como os padres, presbíteros e pastores).

Contada, resumidamente, através do significado destas três palavras, aqui está a explicação porque
somos chamados doutores! Porque a Medicina, em nossa tradição, é um sacerdócio, uma atividade sagrada!
Deste modo, no mundo cristão, \ingles{physician} se tornou doutor! Pronto! Este é o cerne,
o ponto central desta aula. O mais importante que devemos aprender. Aquilo que não podemos esquecer.

Em ambas as tradições --- e isto está implícito tanto na palavra \ingles{physician} como na palavra
doutor --- há uma grande responsabilidade para o médico e a Medicina tem, quer gostemos, quer não
gostemos, uma implicação também religiosa. Não me proponho a discutir estas implicações religiosas,
neste momento. Seria me alongar demais. No entanto, se coincidem nisto, as tradições grega e bíblica
diferem profundamente quanto ao papel e quanto ao tipo de responsabilidade do médico.

Para a tradição bíblica, o médico é mais do
que um cientista ou um técnico, é mais do que um entendido em \ingles{phisys} (natureza) e sua
responsabilidade não é total. O homem não é um deus e, portanto, não pode ter responsabilidade total.
Ele erra. O médico erra. O médico, freqüentemente, não conhece e, muitas vezes, quando conhece não
dispõe de todos os recursos.

Não. Nesta tradição, a responsabilidade final de todas as coisas é de Deus. A nossa é uma
responsabilidade compartilhada por varios agentes: médicos, pacientes (sim, o paciente também é
responsável), familiares, sociedade, governo. O que se espera do médico é que faça o seu melhor.
Não mais. E nas condições em que está inserido. Dentro do possível. Não se pode exigir sempre o melhor
resultado. Quem pode assegurar que um diagnóstico é 100\% correto? Que o tratamento será 100\% eficaz?
Que não haverá complicações? Que o doente não morrerá? Não, o médico não é deus! Ele tem o dever
de fazer seu melhor.

Cumpre-lhe assegurar cuidado, buscar entendimento, ensinar ao paciente e uma fazer a intervenção
possível. Exige-se do médico um cuidado humano, pessoal e uma adequada aplicação dos recursos. Mas,
se não há cura ou se o paciente morre, isto não significa, necessariamente, como pensavam os gregos,
um ato criminoso do médico. Algo que, podendo ser feito, não foi feito ou algo que foi feito errado.

A responsabilidade do médico é de meios e não de fins. O médico não pode assegurar resultados.
Ele tem o dever de aprimorar seus meios de intervenção sobre a doença. Tem a obrigação de estudar
e de ensinar. A ênfase é no aprendizado, no aprimoramento, no fazer o melhor. Não no controle.
Não em garantir um resultado. O controle e o resultado, em seus determinantes últimos,
dependem de Deus. Não do médico.

Estes dois sentidos, expressos nas plavras \ingles{physician} e doutor (e suas implicações),
relacionam-se com a situação presente. Hoje, vivemos tempos difíceis, tempos esquizofrênicos.
Vocês serão vistos tanto como \ingles{physicians} (controladores) como doutores (ensinadores).

Oxalá, todos os pacientes procurem vocês como doutores. Não exijam milagres. Sejam responsáveis,
colaborem para sua própria cura. Aos que procurarem vocês como \ingles{physicians}, recomendo que
façam milagres! Porque, nada menos que isto os satisfarão. Bem, talvez, alguns fiquem satisfeitos
com uma grande indenização!

Isto, evidentemente, é uma brincadeira. Ou até um desabafo de um doutor com 21 anos de experiência.
Pensem no que se exige hoje da Dermatologia e da Cirurgia Plástica. Ou nos processos por erro médico?
Um patologista americano pagou (através de sua seguradora, evidentemente) cerca de 7 milhões de
dólares por um erro de diagnóstico de câncer! Como não errar? Como garantir a cura? Como fazer
milagres? Eu não sei. Sou apenas um doutor.

Nossos pacientes esperam muito de nós. Confrontados com tão grande expectativa, como satisfazê-la?
Evidentemente, há muitas causas e explicações para esta elevação da expectativa em relação à Medicina.
Eu me refiro a essa idéia perniciosa que a Medicina pode resolver todos os problemas de saúde;
que todos os recursos devem estar disponíveis para todos, todo o tempo e em todo o lugar; que há um
remédio ou cirurgia para tudo; que, quando não se obtém a completa cura, é culpa de alguém,
quase sempre do médico.

Como disse, há muitas explicações para esta situação e, não cabe, neste momento, discuti-las.
Direi, apenas, que, no meu entender, esta idéia perniciosa resulta da repaganização da nossa sociedade.
Por isso há uma demanda por \ingles{physician}. Afinal, se conhecemos o genoma humano, se podemos clonar!
Quando nada mais puder resolver, basta apelar para que um Dr.\ Albieri --- esta ridícula personagem da novela
``O Clone'' --- faça um clone, um outro eu. Bem, nem tudo está perdido: afinal, a Jade é quem faz mais sucesso!

Enquanto isto, os doutores são vilipendiados. Perderam prestígio, ganham pouco ou até ganham bem,
mas ao custo de excessivo trabalho.

Sim, um \ingles{physician} pode tudo, promete resultado! Um doutor faz, apenas, o melhor possível.
Um doutor ensina seus pacientes. Diz-lhes a verdade. No consultório, olho no olho. Não na mídia.
Obtém sua colaboração para alcançar a saúde possível. Ajuda seus pacientes a aceitar a dura realidade.
A realidade da condição humana. As nossas limitações: dos médicos, dos pacientes, de todos os homens
e também do governo, dos planos de saúde, dos hospitais, dos administradores de saúde.

Nossa condição inclui a doença, o sofrimento e a morte. Como doutores, quase sempre podemos minorar
o sofrimento, sempre podemos confortar e ajudar a entender, algumas vezes podemos adiar um pouco a morte
e até podemos curar. Mas somos apenas doutores. Não somos deuses. Não fazemos milagres!

Chegamos ao final da aula. Eu pretendi ensinar o que é ser médico através do significado de duas palavras.
Explicar o conflito entre \ingles{physician} e doutor. Está me parecendo que fiz mais um discurso do
que dei uma aula. De qualquer modo, resta-me, agora, para encerrar, fazer o meu pedido para os novos doutores.

Assim, concluída a aula, eis o meu pedido.
Bem, talvez este pedido seja também um conselho. Daqueles conselhos que pais, padres, pastores, médicos,
isto é, doutores no mais antigo sentido, costumam dar. E para os quais, muitas vezes, torcemos o nariz,
porque sabemos que vai sobrar para nós. Que vai nos dizer que também somos responsáveis!

Meu pedido, como paraninfo de vocês é o seguinte: sejam fiéis à vocação que escolheram. Sejam doutores!
O começo é agora. Hoje, vocês se tornaram doutores. Não prometam milagres. Façam o seu melhor.

Eu vos felicito, doutores.

Muito obrigado.

\vspace{0.5 cm}

Paulo Roberto Merçon de Vargas


%%%%%%%%%%%%%%%%%%%%%%%%%%%%%%%%%%%%%%%%%%%%%%%%%%%%%%%%%%%%%%%%%%%%%%%%%%%%%%%%%
%%%%%%%%%%%%%%%%%%%%%%%%%%%%%%%%%%%%%%%%%%%%%%%%%%%%%%%%%%%%%%%%%%%%%%%%%%%%%%%%%
%%%%%%%%%%%%%%%%%%%%%%%%%%%%%%%%%%%%%%%%%%%%%%%%%%%%%%%%%%%%%%%%%%%%%%%%%%%%%%%%%
%\section{}
%\label{}


%%%%%%%%%%%%%%%%%%%%%%%%%%%%%%%%%%%%%%%%%%%%%%%%%%%%%%%%%%%%%%%%%%%%%%%%%%%%%%%%%
%%%%%%%%%%%%%%%%%%%%%%%%%%%%%%%%%%%%%%%%%%%%%%%%%%%%%%%%%%%%%%%%%%%%%%%%%%%%%%%%%
%\subsection{}
%\label{}

%%%%%%%%%%%%%%%%%%%%%%%%%%%%%%%%%%%%%%%%%%%%%%%%%%%%%%%%%%%%%%%%%%%%%%%%%%%%%%%%%
%\subsubsection{}
%\label{}




%%%%%%%%%%%%%%%%%%%%%%%%%%%%%%%%%%%%%%%%%%%%%%%%%%%%%%%%%%%%%%%%%%%%%%%%%%%%%%%%%
%%%%%%%%%%%%%%%%%%%%%%%%%%%%%%%%%%%%%%%%%%%%%%%%%%%%%%%%%%%%%%%%%%%%%%%%%%%%%%%%%
%%%%%%%%%%%%%%%%%%%%%%%%%%%%%%%%%%%%%%%%%%%%%%%%%%%%%%%%%%%%%%%%%%%%%%%%%%%%%%%%%
%%%%%%%%%%%%%%%%%%%%%%%%%%%%%%%%%%%%%%%%%%%%%%%%%%%%%%%%%%%%%%%%%%%%%%%%%%%%%%%%%
%%%%%%%%%%%%%%%%%%%%%%%%%%%%%% TERMINA O DOCUMENTO %%%%%%%%%%%%%%%%%%%%%%%%%%%%%%
%%%%%%%%%%%%%%%%%%%%%%%%%%%%%%%%%%%%%%%%%%%%%%%%%%%%%%%%%%%%%%%%%%%%%%%%%%%%%%%%%
%%%%%%%%%%%%%%%%%%%%%%%%%%%%%%%%%%%%%%%%%%%%%%%%%%%%%%%%%%%%%%%%%%%%%%%%%%%%%%%%%
%%%%%%%%%%%%%%%%%%%%%%%%%%%%%%%%%%%%%%%%%%%%%%%%%%%%%%%%%%%%%%%%%%%%%%%%%%%%%%%%%
%%%%%%%%%%%%%%%%%%%%%%%%%%%%%%%%%%%%%%%%%%%%%%%%%%%%%%%%%%%%%%%%%%%%%%%%%%%%%%%%%
\end{document}
